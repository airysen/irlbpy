%\documentclass[landscape,a0b,final,a4resizeable]{a0poster}
\documentclass[landscape,a0b,final]{a0poster}
%\documentclass[portrait,a0b,final,a4resizeable]{a0poster}
%\documentclass[portrait,a0b,final]{a0poster}
%%% Option "a4resizeable" makes it possible ot resize the
%   poster by the command: psresize -pa4 poster.ps poster-a4.ps
%   For final printing, please remove option "a4resizeable" !!
\usepackage{epsfig}
\usepackage{multicol}
\usepackage{pstricks,pst-grad}



%%%%%%%%%%%%%%%%%%%%%%%%%%%%%%%%%%%%%%%%%%%
% Definition of some variables and colors
%\renewcommand{\rho}{\varrho}
%\renewcommand{\phi}{\varphi}
\setlength{\columnsep}{3cm}
\setlength{\columnseprule}{2mm}
\setlength{\parindent}{0.0cm}



%%%%%%%%%%%%%%%%%%%%%%%%%%%%%%%%%%%%%%%%%%%%%%%%%%%%
%%%               Background                     %%%
%%%%%%%%%%%%%%%%%%%%%%%%%%%%%%%%%%%%%%%%%%%%%%%%%%%%

\newcommand{\background}[3]{
  \newrgbcolor{cgradbegin}{#1}
  \newrgbcolor{cgradend}{#2}
  \psframe[fillstyle=gradient,gradend=cgradend,
  gradbegin=cgradbegin,gradmidpoint=#3](0.,0.)(1.\textwidth,-1.\textheight)
}



%%%%%%%%%%%%%%%%%%%%%%%%%%%%%%%%%%%%%%%%%%%%%%%%%%%%
%%%                Poster                        %%%
%%%%%%%%%%%%%%%%%%%%%%%%%%%%%%%%%%%%%%%%%%%%%%%%%%%%

\newenvironment{poster}{
  \begin{center}
  \begin{minipage}[c]{0.98\textwidth}
}{
  \end{minipage} 
  \end{center}
}



%%%%%%%%%%%%%%%%%%%%%%%%%%%%%%%%%%%%%%%%%%%%%%%%%%%%
%%%                pcolumn                       %%%
%%%%%%%%%%%%%%%%%%%%%%%%%%%%%%%%%%%%%%%%%%%%%%%%%%%%

\newenvironment{pcolumn}[1]{
  \begin{minipage}{#1\textwidth}
  \begin{center}
}{
  \end{center}
  \end{minipage}
}



%%%%%%%%%%%%%%%%%%%%%%%%%%%%%%%%%%%%%%%%%%%%%%%%%%%%
%%%                pbox                          %%%
%%%%%%%%%%%%%%%%%%%%%%%%%%%%%%%%%%%%%%%%%%%%%%%%%%%%

\newrgbcolor{lcolor}{0. 0. 0.80}
\newrgbcolor{gcolor1}{1. 1. 1.}
\newrgbcolor{gcolor2}{.80 .80 1.}

\newcommand{\pbox}[4]{
\psshadowbox[#3]{
\begin{minipage}[t][#2][t]{#1}
#4
\end{minipage}
}}



%%%%%%%%%%%%%%%%%%%%%%%%%%%%%%%%%%%%%%%%%%%%%%%%%%%%
%%%                myfig                         %%%
%%%%%%%%%%%%%%%%%%%%%%%%%%%%%%%%%%%%%%%%%%%%%%%%%%%%
% \myfig - replacement for \figure
% necessary, since in multicol-environment 
% \figure won't work

\newcommand{\myfig}[3][0]{
\begin{center}
  \vspace{1.5cm}
  \includegraphics[width=#3\hsize,angle=#1]{#2}
  \nobreak\medskip
\end{center}}



%%%%%%%%%%%%%%%%%%%%%%%%%%%%%%%%%%%%%%%%%%%%%%%%%%%%
%%%                mycaption                     %%%
%%%%%%%%%%%%%%%%%%%%%%%%%%%%%%%%%%%%%%%%%%%%%%%%%%%%
% \mycaption - replacement for \caption
% necessary, since in multicol-environment \figure and
% therefore \caption won't work

%\newcounter{figure}
\setcounter{figure}{1}
\newcommand{\mycaption}[1]{
  \vspace{0.5cm}
  \begin{quote}
    {{\sc Figure} \arabic{figure}: #1}
  \end{quote}
  \vspace{1cm}
  \stepcounter{figure}
}



%%%%%%%%%%%%%%%%%%%%%%%%%%%%%%%%%%%%%%%%%%%%%%%%%%%%%%%%%%%%%%%%%%%%%%
%%% Begin of Document
%%%%%%%%%%%%%%%%%%%%%%%%%%%%%%%%%%%%%%%%%%%%%%%%%%%%%%%%%%%%%%%%%%%%%%

\begin{document}

\background{1. 1. 1.}{1. 1. 1.}{0.5}

\vspace*{2cm}


\newrgbcolor{lightblue}{0. 0. 0.80}
\newrgbcolor{white}{1. 1. 1.}
\newrgbcolor{whiteblue}{.80 .80 1.}


\begin{poster}

%%%%%%%%%%%%%%%%%%%%%
%%% Header
%%%%%%%%%%%%%%%%%%%%%
\begin{center}
\begin{pcolumn}{0.98}

\pbox{0.95\textwidth}{}{linewidth=2mm,framearc=0.3,fillstyle=gradient,gradangle=0,gradbegin=white,gradend=white,gradmidpoint=1.0,framesep=1em}{

\begin{center}
  {\sc \Huge irlbpy -- A fast partial SVD for Python}\\[10mm]
  {\Large J. Baglama$\ ^1$, M. Kane$\ ^2$, B. Lewis$\ ^3$, and L. Reichel$\ ^4$\\[7.5mm]
  $\ ^1$Department of Mathematics, The Unversity of Rhode Island\\
  $\ ^2$Department of Biostatistics, Yale University\\
  $\ ^3$Paradigm4\\
  $\ ^4$Department of Mathematics, Kent State University}
\end{center}
}
\end{pcolumn}
\end{center}

\vspace*{2cm}

%%% Begin of Multicols-Enviroment
\begin{multicols}{3}

%%% Overview
\begin{center}\pbox{0.8\columnwidth}{}{linewidth=2mm,framearc=0.1,fillstyle=gradient,gradangle=0,gradbegin=white,gradend=white,gradmidpoint=1.0,framesep=1em}{\begin{center}\bf{Overview}\end{center}}\end{center}
\vspace{1.25cm}

The singular value decomposition (SVD) is central to many important analysis
methods and applications including principal component analysis, canonical
correlation analysis, correspondence analysis, latent semantic indexing and
non-linear iterative partial least squares to name a few. However, numerical 
implementations of the SVD are computationally intensive, generally incurring
a computational complexity of $O(m^2n + n^3)$ for an $m \times n$ matrix with
$m$ greater than $n$. As a result, data scientist's have fewer analytical 
tools to understand the structure of data as those data become large and
the resulting computational cost becomes too expensive to carry out.

\vspace{0.75cm}

However, many of these methods and applications only require a few singular
values and corresponding singular vectors. With this in mind, some researchers
have focused on computational efficient {\em truncated} SVD algorithm that
calculates the largest or smallest singular value information for a matrix.
The {\em implicitly restarted Lanczos bidiagonalization} (IRLB) algorithm
\cite{Baglama2006} is a fast and efficient approach
for calculating truncated singular values, generally scaling linearly in the
size of the matrix. This innovative approach to calculating a key numerical
decomposition for statistical and machine learning procedures allows many
standard analyses to scale to much larger data sets than previously possible
and suggests new approximation algorithms where current approaches fall
short.

\vspace{0.75cm}

While irlb implementations have existed for some time now in both 
Matlab \cite{irlbMatlab} and R \cite{irlbR} the scientific Python community
has not enjoyed the computational savings offered by the algorithm until now.
This poster introduces the irlbpy package for Python, a pip-installable 
open-source implementation of the IRLB algorithm that is available from
github at https://github.com/bwlewis/irlbpy. The package leverages
numpy for dense matrices as well as scipy for sparse matrices. The rest
of this poster gives an overview of the algorithm and benchmarks performance.
The actual benchmarks were performed on a Mac Book Pro with a quad-core
2.7 GHz Intel Core i7 with 16 GB of 1600 MHz DDR3 RAM running Python 
version 2.7.3, Numpy version 1.7.0, and SciPy version 0.12.0.

\begin{enumerate}
\item introduce the irbpy package
\item reference irlba R, and Matlab implementation
\item works with sparse and dense matrices
\item give the github address
\item give description of machine where benchmarks were done
\end{enumerate}


%Many applications of the SVD often require only a
%few singular values and corresponding singular vectors. We've written a
%prototype Scikit implementations of the Baglama-Reichel implicitly-restarted
%Lanczos (IRLB) methods for computing a few singular values and corresponding
%vectors of a matrix.  The methods are compatible with dense or scipy.sparse
%matrices.  The IRBL method significantly outperforms existing SVD
%implementations in computational and memory efficiency.  Adaptaion of IRLB to
%parallel computation for very large problems is straightforward.

\columnbreak

%%% Introduction
\vspace{2cm}\begin{center}\pbox{0.8\columnwidth}{}{linewidth=2mm,framearc=0.1,fillstyle=gradient,gradangle=0,gradbegin=white,gradend=white,gradmidpoint=1.0,framesep=1em}{\begin{center}\bf{Mathematical Approach}\end{center}}\end{center}\vspace{1.25cm}

\columnbreak

%%% Section
\vspace{2cm}\begin{center}\pbox{0.8\columnwidth}{}{linewidth=2mm,framearc=0.1,fillstyle=gradient,gradangle=0,gradbegin=white,gradend=white,gradmidpoint=1.0,framesep=1em}{\begin{center}\bf{Algorithm Implementation}\end{center}}\end{center}\vspace{1.25cm}


%% References
\bibliographystyle{alpha}
\bibliography{poster.bib}

\end{multicols}

\vspace{1in}

\begin{multicols}{3}
%%% Abstract

\vspace{1in}

\vspace{2cm}\begin{center}\pbox{0.8\columnwidth}{}{linewidth=1mm,framearc=0.1,fillstyle=gradient,gradangle=0,gradbegin=white,gradend=white,gradmidpoint=1.0,framesep=1em}{\begin{center}\bf{Dense Matrix Comparison}\end{center}}\end{center}\vspace{1.25cm}

\begin{center}
  % first argument: eps-file
  % second argument: stretching-factor relative to Column-width (<1)
  % optional argument: rotation angle (0-360), default=0
  \myfig[0]{figures/DenseMatrixComparison.eps}{1}
  \mycaption{Perfomance comparison of the IRLB and the numpy implementation
of the SVD.}
\end{center}

\columnbreak

\vspace{1in}

\vspace{2cm}\begin{center}\pbox{0.8\columnwidth}{}{linewidth=1mm,framearc=0.1,fillstyle=gradient,gradangle=0,gradbegin=white,gradend=white,gradmidpoint=1.0,framesep=1em}{\begin{center}\bf{Dense Matrix Scaling}\end{center}}\end{center}\vspace{1.25cm}

\begin{center}
  \myfig[0]{figures/DenseMatrixNuScaling.eps}{1}
  \mycaption{The time required to calculate the IRLB on dense matrices for 
specified values of nu (the number of singular vectors).}
\end{center}

\columnbreak

\vspace{1in}

\vspace{2cm}\begin{center}\pbox{0.8\columnwidth}{}{linewidth=1mm,framearc=0.1,fillstyle=gradient,gradangle=0,gradbegin=white,gradend=white,gradmidpoint=1.0,framesep=1em}{\begin{center}\bf{Sparse Matrix Scaling} \end{center}}\end{center}\vspace{1.25cm}

\begin{center}
  % first argument: eps-file
  % second argument: stretching-factor relative to Column-width (<1)
  % optional argument: rotation angle (0-360), default=0
  \myfig[0]{figures/SparseMatrixNuScaling.eps}{1}
  \mycaption{The time required to calculate the IRLB on sparse matrices for 
specified values of nu (the number of singular vectors).}
\end{center}

\end{multicols}

\end{poster}

\end{document}

