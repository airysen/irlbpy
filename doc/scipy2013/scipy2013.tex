%\documentclass[landscape,a0b,final,a4resizeable]{a0poster}
\documentclass[landscape,a0b,final]{a0poster}
%\documentclass[portrait,a0b,final,a4resizeable]{a0poster}
%\documentclass[portrait,a0b,final]{a0poster}
%%% Option "a4resizeable" makes it possible ot resize the
%   poster by the command: psresize -pa4 poster.ps poster-a4.ps
%   For final printing, please remove option "a4resizeable" !!
\usepackage{epsfig}
\usepackage{multicol}
\usepackage{pstricks,pst-grad}



%%%%%%%%%%%%%%%%%%%%%%%%%%%%%%%%%%%%%%%%%%%
% Definition of some variables and colors
%\renewcommand{\rho}{\varrho}
%\renewcommand{\phi}{\varphi}
\setlength{\columnsep}{3cm}
\setlength{\columnseprule}{2mm}
\setlength{\parindent}{0.0cm}



%%%%%%%%%%%%%%%%%%%%%%%%%%%%%%%%%%%%%%%%%%%%%%%%%%%%
%%%               Background                     %%%
%%%%%%%%%%%%%%%%%%%%%%%%%%%%%%%%%%%%%%%%%%%%%%%%%%%%

\newcommand{\background}[3]{
  \newrgbcolor{cgradbegin}{#1}
  \newrgbcolor{cgradend}{#2}
  \psframe[fillstyle=gradient,gradend=cgradend,
  gradbegin=cgradbegin,gradmidpoint=#3](0.,0.)(1.\textwidth,-1.\textheight)
}



%%%%%%%%%%%%%%%%%%%%%%%%%%%%%%%%%%%%%%%%%%%%%%%%%%%%
%%%                Poster                        %%%
%%%%%%%%%%%%%%%%%%%%%%%%%%%%%%%%%%%%%%%%%%%%%%%%%%%%

\newenvironment{poster}{
  \begin{center}
  \begin{minipage}[c]{0.98\textwidth}
}{
  \end{minipage} 
  \end{center}
}



%%%%%%%%%%%%%%%%%%%%%%%%%%%%%%%%%%%%%%%%%%%%%%%%%%%%
%%%                pcolumn                       %%%
%%%%%%%%%%%%%%%%%%%%%%%%%%%%%%%%%%%%%%%%%%%%%%%%%%%%

\newenvironment{pcolumn}[1]{
  \begin{minipage}{#1\textwidth}
  \begin{center}
}{
  \end{center}
  \end{minipage}
}



%%%%%%%%%%%%%%%%%%%%%%%%%%%%%%%%%%%%%%%%%%%%%%%%%%%%
%%%                pbox                          %%%
%%%%%%%%%%%%%%%%%%%%%%%%%%%%%%%%%%%%%%%%%%%%%%%%%%%%

\newrgbcolor{lcolor}{0. 0. 0.80}
\newrgbcolor{gcolor1}{1. 1. 1.}
\newrgbcolor{gcolor2}{.80 .80 1.}

\newcommand{\pbox}[4]{
\psshadowbox[#3]{
\begin{minipage}[t][#2][t]{#1}
#4
\end{minipage}
}}



%%%%%%%%%%%%%%%%%%%%%%%%%%%%%%%%%%%%%%%%%%%%%%%%%%%%
%%%                myfig                         %%%
%%%%%%%%%%%%%%%%%%%%%%%%%%%%%%%%%%%%%%%%%%%%%%%%%%%%
% \myfig - replacement for \figure
% necessary, since in multicol-environment 
% \figure won't work

\newcommand{\myfig}[3][0]{
\begin{center}
  \vspace{1.5cm}
  \includegraphics[width=#3\hsize,angle=#1]{#2}
  \nobreak\medskip
\end{center}}



%%%%%%%%%%%%%%%%%%%%%%%%%%%%%%%%%%%%%%%%%%%%%%%%%%%%
%%%                mycaption                     %%%
%%%%%%%%%%%%%%%%%%%%%%%%%%%%%%%%%%%%%%%%%%%%%%%%%%%%
% \mycaption - replacement for \caption
% necessary, since in multicol-environment \figure and
% therefore \caption won't work

%\newcounter{figure}
\setcounter{figure}{1}
\newcommand{\mycaption}[1]{
  \vspace{0.5cm}
  \begin{quote}
    {{\sc Figure} \arabic{figure}: #1}
  \end{quote}
  \vspace{1cm}
  \stepcounter{figure}
}



%%%%%%%%%%%%%%%%%%%%%%%%%%%%%%%%%%%%%%%%%%%%%%%%%%%%%%%%%%%%%%%%%%%%%%
%%% Begin of Document
%%%%%%%%%%%%%%%%%%%%%%%%%%%%%%%%%%%%%%%%%%%%%%%%%%%%%%%%%%%%%%%%%%%%%%

\begin{document}

\background{1. 1. 1.}{1. 1. 1.}{0.5}

\vspace*{2cm}


\newrgbcolor{lightblue}{0. 0. 0.80}
\newrgbcolor{white}{1. 1. 1.}
\newrgbcolor{whiteblue}{.80 .80 1.}


\begin{poster}

%%%%%%%%%%%%%%%%%%%%%
%%% Header
%%%%%%%%%%%%%%%%%%%%%
\begin{center}
\begin{pcolumn}{0.98}

\pbox{0.95\textwidth}{}{linewidth=2mm,framearc=0.3,fillstyle=gradient,gradangle=0,gradbegin=white,gradend=white,gradmidpoint=1.0,framesep=1em}{

\begin{center}
  {\sc \Huge Irlbpy -- A fast partial SVD for Python}\\[10mm]
  {\Large J. Baglama$\ ^1$, M. Kane$\ ^2$, and B. Lewis$\ ^3$}\\[7.5mm]
  $\ ^1$Department of Mathematics, The University of Rhode Island\\
  $\ ^2$Department of Biostatistics, Yale University\\
  $\ ^3$Paradigm4\\
\end{center}
}
\end{pcolumn}
\end{center}

\vspace*{2cm}

%%% Begin of Multicols-Environment
\begin{multicols}{3}

%%% Overview
\begin{center}\pbox{0.8\columnwidth}{}{linewidth=2mm,framearc=0.1,fillstyle=gradient,gradangle=0,gradbegin=white,gradend=white,gradmidpoint=1.0,framesep=1em}{\begin{center}\bf{Overview}\end{center}}\end{center}
\vspace{1.25cm}

The singular value decomposition (SVD) is central to many important
data-analytic methods and applications including principal component analysis,
canonical correlation analysis, correspondence analysis, latent semantic
indexing and non-linear iterative partial least squares to name a few. However,
numerical implementations of the SVD are intensive, generally incurring a
computational complexity of $O(m^2n + n^3)$ for an $m \times n$ real matrix.
As a result, data scientist's have fewer analytical tools to understand the
structure of data as those data become large and the resulting computational
cost becomes too expensive to carry out.

\vspace{0.75cm}

Many important analysis methods and applications only require a subset of the
largest singular values and corresponding singular vectors. With this in mind,
some numerical analysts have focused on the developement of {\em truncated} SVD
algorithms that calculate the largest or smallest singular values and vectors
for a matrix.  The {\em augmented implicitly restarted Lanczos
bidiagonalization} (IRLB) algorithm \cite{Baglama2006} is a fast and efficient
approach for calculating truncated singular value decompositions, generally
scaling linearly in the size of the matrix. This innovative algorithm
calculates a key numerical decomposition for statistical and machine learning
procedures and it effectively scales to massive data.  Standard analyses can
now be applied to much larger data sets than previously possible.  Furthermore,
the approach used by the IRLB algorithm suggests a general class of new
approximation algorithms that can be used for big-data challenges where current
approaches fall short.

\vspace{0.75cm}

While IRLB implementations have existed for some time now in both Matlab
\cite{irlbaMatlab} and R \cite{irlbaR} the scientific Python community has not
enjoyed the computational savings offered by the algorithm until now.  This
poster introduces the irlbpy package for Python, a pip-installable open-source
implementation of the IRLB algorithm that is available from github at
https://github.com/bwlewis/irlbpy. The package is compatible with dense and
sparse data, accepting either numpy 2D arrays or matrices, or scipy sparse
matrices as input. The rest of this poster gives an overview of the algorithm
and benchmarks performance.  The benchmarks were performed on a Mac Book Pro
with a quad-core 2.7 GHz Intel Core i7 with 16 GB of 1600 MHz DDR3 RAM running
Python version 2.7.3, Numpy version 1.7.0, and SciPy version 0.12.0.

\columnbreak

%%% Introduction
\vspace{2cm}\begin{center}\pbox{0.8\columnwidth}{}{linewidth=2mm,framearc=0.1,fillstyle=gradient,gradangle=0,gradbegin=white,gradend=white,gradmidpoint=1.0,framesep=1em}{\begin{center}\bf{Partial SVD Definition}\end{center}}\end{center}\vspace{1.25cm}

A truncated SVD of a matrix $X \in \mathcal{R}^{m \times n}$ computes the
decomposition

\[
XV = U\Sigma,
\]

where $V$ is an $n\times p$ orthonormal matrix,
$U$ is an $m \times p$ orthonormal matrix, $\Sigma$ is an $p \times p$
diagonal matrix, and $1\le p\le\min(m,n)$.
The diagonal values of $\Sigma$ are called singular values of $X$ and
are nonnegative and ordered,
$\sigma_1\ge \sigma_2\ge \cdots \sigma_p \geq 0$.
The columns of $U$ and $V$ are
called the left and right signular vectors, repsectively.

\vspace{0.75cm}

When $p=\min(m,n)$, the truncated singular value decomposition coincides
with the usual singular value decomposition and $A = U\Sigma V^T$. When
$p<\min(m,n)$, then $U\Sigma V^T$ only approximates the matrix $A$.

\vspace{0.75cm}
The irlb algorithm estimates a truncated SVD of X, computing
\[
XV \approx U\Sigma
\]
by projecting the matrix into an augmented sequence of Krylov subspaces.
The quality of the approximation is user-controlled by a subspace iteration
tolerance parameter. The method often converges to a solution quickly even
when the tolerance is set to a small value. The availability of the parameter
allows users to make very quick estimates of a few singular values and
their corresponding vectors, even for large problems.


\columnbreak

%%% Section
\vspace{2cm}\begin{center}\pbox{0.8\columnwidth}{}{linewidth=2mm,framearc=0.1,fillstyle=gradient,gradangle=0,gradbegin=white,gradend=white,gradmidpoint=1.0,framesep=1em}{\begin{center}\bf{Discussion}\end{center}}\end{center}\vspace{1.25cm}

\begin{enumerate}
\item In practice the IRLB algorithm scales linearly in the size of the data.
This gives it a tremendous advantage over the traditional SVD methods when only
the largest singular value information is needed.
\item The algorithms computational cost does increase
as the number of singular values required increases.
\item The IRLB algorithm and irlbpy implementation are compatible with 
both dense numpy matrices and sparse scipy matrices.
\item A distributed version of the IRLB algorithm is availablle
for SciDB \cite{scidb} (and soon indirectly for Python through the
SciDBPy package under development). This version has computed a truncated SVD
on sparse matrices with 50 million rows, 40 million columns and
4 billion non-zero elements on a cluster of 4 dual-Xeon machines
in approximately five minutes per singular value/vector.
\item The irlbpy package can compute a few singular vectors and
corresponding singular values of problems similar to the Netflix
prize problem--a sparse $500000\times18000$ matrix with about $100$ million nonzero entries--in a few minutes on a laptop.
\end{enumerate}

We note that Martinsson, Rokhlin, and others have developed fast truncated
SVD and related decomposition methods using randomized methods, see
for example:
Edo Liberty, Franco Woolfe, Per-Gunnar Martinsson, Vladimir Rokhlin, and Mark Tygert, Randomized algorithms for the low-rank approximation of matrices, Proceedings of the National Academy of Sciences 2007 104: 20167-20172.
These methods are indeed competetive with the IRLB approach, and problems
exist where each outperforms the other. 

%% References
\bibliographystyle{alpha}
\bibliography{poster.bib}

\end{multicols}

\vspace{1in}

\begin{multicols}{3}
%%% Abstract

\vspace{2cm}\begin{center}\pbox{0.8\columnwidth}{}{linewidth=1mm,framearc=0.1,fillstyle=gradient,gradangle=0,gradbegin=white,gradend=white,gradmidpoint=1.0,framesep=1em}{\begin{center}\bf{Dense Matrix Comparison}\end{center}}\end{center}\vspace{1.25cm}

\vspace{-1.25cm}

\begin{center}
  % first argument: eps-file
  % second argument: stretching-factor relative to Column-width (<1)
  % optional argument: rotation angle (0-360), default=0
  \myfig[0]{figures/DenseMatrixComparison.eps}{1}
  \mycaption{Performance comparison of the IRLB and the numpy implementation
of the SVD calculating the 10 larges singular values and vectors.}
\end{center}

\columnbreak

\vspace{2cm}\begin{center}\pbox{0.8\columnwidth}{}{linewidth=1mm,framearc=0.1,fillstyle=gradient,gradangle=0,gradbegin=white,gradend=white,gradmidpoint=1.0,framesep=1em}{\begin{center}\bf{Dense Matrix Scaling}\end{center}}\end{center}\vspace{1.25cm}

\vspace{-1.25cm}

\begin{center}
  \myfig[0]{figures/DenseMatrixNuScaling.eps}{1}
  \mycaption{The time required to calculate the IRLB on dense matrices for 
specified values of nu (the number of singular vectors).}
\end{center}

\columnbreak

\vspace{2cm}\begin{center}\pbox{0.8\columnwidth}{}{linewidth=1mm,framearc=0.1,fillstyle=gradient,gradangle=0,gradbegin=white,gradend=white,gradmidpoint=1.0,framesep=1em}{\begin{center}\bf{Sparse Matrix Scaling} \end{center}}\end{center}\vspace{1.25cm}

\vspace{-1.5cm}

\begin{center}
  % first argument: eps-file
  % second argument: stretching-factor relative to Column-width (<1)
  % optional argument: rotation angle (0-360), default=0
  \myfig[0]{figures/SparseMatrixNuScaling.eps}{1}
  \mycaption{The time required to calculate the IRLB on sparse matrices for 
specified values of nu (the number of singular vectors).}
\end{center}

\vspace{1cm}

\columnbreak

\end{multicols}

\end{poster}

\end{document}

